\chapter{Circuit-Cocircuit algorithm}
\label{ch:algorithm}

\section{Abstract Circuit-Cocircuit Meta-algorithm}

The problem of exhaustive generation of minimal edge cuts in \linebreak a graph can be generalized to the problem of exhaustive generation of cocircuits in a matroid.

\begin{defn}[Cycle matroid]
	\label{defn:cycle_matroid}
	Given a connected graph $G$ its \textit{cycle matroid} is a matroid on the ground set $E(G)$ and its bases are the edges of the spanning trees of G. Moreover independent sets are the acylic subsets (forests) of G, circuits are the cycles of G and cocircuits are the minimal edge cuts of G.
\end{defn}

We will use the following folklore statement, see Oxley \cite{oxley2006matroid}.

\begin{prop}
	\label{prop_circuit-cocircuit}
	If $C$ is a~circuit and $X$ is a cocircuit in a matroid M, then ${\lvert C \cap X \rvert \neq 1}$.
\end{prop}

Using this proposition an algorithm with the following gist can be constructed. Start with any element of $M$ in $X$, find a circuit $C$, such that \linebreak ${\lvert C \cap X \rvert = 1}$ and then for each element $c \in C \setminus X$, add $c$ to $X$, verify there is a hyperplane in the complement of $X$ and continue recursively for as long as such circuits can be found.

\clearpage

\begin{algorithm}
	\caption{Abstract Circuit-Cocircuit Meta-algorithm}
	\label{alg:meta}
\begin{algorithmic}[1]
	\Require Matroid $M = (E, \mathcal{B})$ and a cocircuit size bound $m \in \mathbb{N}$
	\Ensure All cocircuits of $M$ with size $\leq m$
	\State $B \leftarrow$ an arbitrary basis in $\mathcal{B}$
	\ForAll{$b \in B$}
		\State $X \leftarrow \{b\}$
		\State \Call{GenCocircuits}{$X$}
	\EndFor
	\Procedure{GenCocircuits}{$X$}
	\If{$E \setminus X$ contains no hyperlane of $M$ \textbf{or} $\lvert X \rvert > m$}
		\State \textbf{return $\undefined$}
	\EndIf
	\State Find any circuit $C \subseteq E$ such that $\lvert C \cap X \rvert = 1$
	\If{such $C$ doesn't exist}
		\State \textbf{return} $X$ \Comment{$X$ is a cocircuit}
	\Else
		\State $D \leftarrow C \setminus X$
		\ForAll{$c \in D$}
			\State \Call{GenCocircuits}{$X \cup \{c\}$}
		\EndFor
	\EndIf

	\EndProcedure
\end{algorithmic}
\end{algorithm}


\begin{thm}
	\label{meta_alg_correctness}
	Algorithm \ref{alg:meta} generates exactly all the cocircuits of size $\leq m$ in the matroid $M$.
\end{thm}

\begin{proof}
	First, we show that any set $X_0$ returned by the algorithm is a~cocircuit and $\lvert X_0 \rvert \leq m$. By the condition on line 7, the set $E \setminus X_0$ contains a~hyperplane $H$ of $M$ and $\lvert X_0 \rvert \leq m$. If $H = E \setminus X_0$ then we are done by Claim \ref{matroid_folklore}.c. Now suppose $H \neq E \setminus X_0$ and choose $e \in E \setminus (X_0 \cup H)$. According to Claim \ref{matroid_folklore}.b there is a basis $B_0 \subseteq H \cup \{e\}$. Choose $a \in X_0$ and let $C_0$ be the circuit contained in $B_0 \cup \{a\}$ according to Claim \ref{matroid_folklore}.a. Then $C_0 \cap X_0 = \{a\}$ contradicts the condition on line 11 and thus $H$ must be equal to $E \setminus X_0$.

	Second, we show that any cocircuit $X_1$, $\lvert X_1 \rvert \leq m$, is returned. Let ${Z_1 \subseteq X_1}$ be the largest subset of $X_1$ such that GenCocircuits($Z_1$) has been called. By Claim \ref{matroid_folklore}.d, $X_1 \cap B$ is nonempty for the basis $B$ selected on line $1$ and thus $Z_1$ is well-defined and non-empty.
	If ${Z_1 = X_1}$ then $X_1$ will be returned since it is a cocircuit and no circuit $C$ is found on line $10$. Suppose $Z_1 \neq X_1$ and choose $d \in X_1 \setminus Z_1$. By Claim \ref{matroid_folklore}.b there is a basis $B_1 \subseteq H_1 \cup \{d\}$ (where $H_1 := E \setminus X_1$ is a hyperplane of $M$). For any $b \in Z_1$, let $C_1$ be the circuit contained in $B_1 \cup \{b\}$ according to Claim \ref{matroid_folklore}.a. Consequently, there exists a~circuit $C$ to be found by the algorithm on line 10 (note that it may be $C \neq C_1$). Since $\lvert C \cap X_1 \rvert \neq 1$, there exists $d' \in C \cap X_1$ such that $d'\not\in Z_1$, and \textsc{GenCocircuits}($Z_1 \cup \{d'\}$) is eventually called, contradicting the maximality of $Z_1$.
\end{proof}

For some $Z_1$ the set $D$ has to be chosen (on line 14) such that for any cocircuit $X_1 \supset Z_1$, $\lvert X_1 \rvert \leq m$, the intersection $D \cap X_1$ is nonempty and thus contains an element to extend $Z_1$.

\begin{cor}
	\label{cor_circuit_subset}
	Algorithm \ref{alg:meta} remains correct even if line 14 sets $D$ to be a subset of $C \setminus X$, such that $D$ is guaranteed to intersect every cocircuit $X_1 \supseteq X$ in $M$.
\end{cor}

% TODO: V poznamkach je vyznaceno "short proof", ale prijde mi, ze je to primo v dukazu predchozi vety; "there exists d' \in C \cap X_1 such that d' \not \in Z_1 and GenCocircuits(Z_1 + d') is called"

\begin{rem}
	The algorithm does nondeterministic choices on lines 1, 10, 15 and may also be nondeterministic on line 14 by Corollary \ref{cor_circuit_subset}.
\end{rem}

\section{$2$-way cuts in a graph}

We present details of implementing the Meta-algorithm to generate $2$-way cuts in a given graph.

We introduce a colouring of the graph (in the sense of labeling) to help us decide the hyperplane existence question on line 7. During the progress of the algorithm some edge $e$ is chosen to be added to $X$ and the algorithm continues to call \textsc{GenCocircuits}. Before this call we colour one end of $e$ red and the other blue. When adding the first edge $e = \{u ,v\}$ to $X$ it is arbitrary whether $u$ is coloured red and $v$ blue (or vice versa). The situation with the next edges added to $X$ is described below. This colouring has to be consistent among edges sharing a vertex.

Let $V(X) = V_r \cup V_b$ and call $V_r$ the red vertices of $X$ and $V_b$ the blue vertices of $X$. We maintain a red tree $T_r$ interconnecting $V_r$. Call ${T_b \subseteq (G \setminus T_r \setminus X)}$ the blue tree interconnecting $V_b$. It will turn out that it is not necessary to store this tree explicitly in a variable.

We add to $T_r$ only those edges $e$ for which it is guaranteed that \textsc{GenCocircuits} is called with $X \cup \{e\}$. We do not remove elements from $T_r$ (within recursive calls) while elements may be removed from $T_b$ and added to $X$. This shows that $T_r$ and $T_b$ are not treated symmetrically. In the end we want to have a cut $X$ separating the graph into two connected components: $T_r$ in one component, $T_b$ in the other. It follows that $V(T_r) \cap V(T_b) = \varnothing$ and $V_r \subseteq V(T_r)$, $V_b \subseteq V(T_b)$.

\begin{lem}
	\label{lem:hyperplane_existence}
	Let $G$ be a~connected graph, $M = (E, \mathcal{B})$ its corresponding cycle matroid and $Y \subseteq E(G)$. $E \setminus Y$ contains a hyperplane if, and only if, the vertices of each edge in $Y$ can be coloured red and blue, such that the vertices of each colour can be connected by a~tree in $G \sm Y$. This colouring has to be consistent among the edges in $Y$.
\end{lem}

\begin{proof}
	$(=>)$
	If $E \sm Y$ contains a hyperplane, then there exists a~cocircuit $X \supseteq Y$ by Claim \ref{matroid_folklore}.c. By Definition \ref{defn:cycle_matroid} $X$ is a bond in $G$ and $G \sm X$ has two connected components, as more components would contradict the minimality of $X$. Colouring one component red and the other blue finishes the argument.

	$(<=)$ Given are two trees, one coloured red and the other blue. Start in one vertex of the red tree and traverse graph $G$ without crossing the to-be generated bond $X \supseteq Y$. Repeat this procedure in the blue tree. This determines some bond $X$, therefore a cocircuit in $M$ and thus by Claim \ref{matroid_folklore}.c a~hyperplane $E \sm X$. Consequently $E \sm Y$ contains a hyperplane since $Y \subseteq X$.
\end{proof}

\begin{cor}
	Let $G$ be a connected graph, $Y \subseteq E(G)$ and let $G$ be coloured according to Lemma \ref{lem:hyperplane_existence}. For any bond $X \subseteq E(G)$ such that $Y$ is a prefix of $X$, the colouring of $G$ can be extended such that each component of $G \sm X$ can be coloured red or blue respectivelly.
\end{cor}

\subsection*{$2$-bonds in a graph}


\begin{rec}
	Let $M$ be the cycle matroid of graph $G$. The bases, circuits, cocircuits of $M$ are the sets of edges of the spanning trees, cycles and bonds in $G$.
\end{rec}

\begin{ams_algorithm}[Circuit-Cocircuit algorithm for 2-bonds]
	\label{alg:2bonds}
	\hfill \\
	Modifying Algorithm \ref{alg:meta}:

	\begin{itemize}
		\item On line 1 choose an arbitrary spanning tree of $G$.

		\item On line 3 the existence of a~hyperplane can be easily decided with use of Lemma \ref{lem:hyperplane_existence} and the previously defined colouring. $E \setminus X$ contains a hyperplane of $M$ if, and only if, $T_r$ and $T_b$ are both connected subgraphs of $G$.

		\item Choose $D$ to be the shortest path in $G \setminus X$ between a vertex from $V(T_r)$ and a vertex from $V_b$.
			For every such $D$, $D \cap E(T_r) = \varnothing$ and there exists a cycle $C \subseteq D \cup E(T_r) \cup X$ in $G$ with $\lvert C \cap X \rvert = 1$.

	\end{itemize}
\end{ams_algorithm}

First two modifications are possible from the definition of cycle matroid. The third modification is possible due to Claim \ref{claim:tr_vb_path}.

\begin{claim}
	\label{claim:tr_vb_path}
	In the setting of Algorithm \ref{alg:meta} let $D$ be the shortest path in $G \setminus X$ between a vertex from $V(T_r)$ and a vertex from $V_b$. Every such $D$ satisfies the requirements of Corollary \ref{cor_circuit_subset}.
\end{claim}


\begin{proof}
	In the base case, $T_r = \varnothing$, $D$ is equal to $C \sm X$.

Now consider an arbitrary recursive call to \textsc{GenCocircuits}. By the induction hypothesis the algorithm has not missed any subset of some $2$-bond yet and by the colouring scheme described above we add to $T_r$ only those edges $e$ for which the algorithm guarantees it has called \textsc{GenCocircuits}$(X \cup \{e\})$.
\end{proof}

\section{$k$-way cuts in graph}

Algorithm \ref{alg:2bonds} generates the $2$-bonds of some connected graph $G$. Now consider $G' = G \setminus Y$, where $Y$ is a $2$-bond of $G$. $G'$ has two connected components and we can apply Algorithm \ref{alg:2bonds} again to each of them to generate the $2$-bonds of the components. It can be proved that the union of $Y$ and any $2$-bond of $G'$ is a $3$-bond of $G$. Such a recursive application of the algorithm can be used to generate all $k$-bonds for some given $k \geq 0$. To formalize this \textit{stepwise} approach and to set it up within the scheme of Meta-algorithm \ref{alg:meta}, we start with the definition of $k$-way cycle matroid.

\begin{defn}[$k$-way cycle matroid]
	\label{defn:kway_cycle_matroid}
	Let $G$ be a graph on less than $k \geq 2$ connected components. The \textit{$k$-way cycle matroid} of $G$ is a matroid on the ground set $E(G)$, such that its bases are the edge sets of the spanning forests of $G$ that consist of $k-1$ trees.

	The bases, circuits, cocircuits and hyperplanes of the $k$-way cycle matroid are also called the $k$-way bases, circuits, cocircuits, hyperplanes of $G$.
\end{defn}

From the definition above we may conclude the following claim.

\begin{claim}
	\label{claim:circuit_types}
	Let $G$ be a graph consisting of less than $k \geq 2$ connected components.
	The $k$-way cocircuits of $G$ are the $k$-bonds of $G$. The $k$-way circuits of $G$ are of two types, \textit{type-C} and \textit{type-F}.

	\begin{enumerate}
		\item Type-C circuits are the graph cycles in $G$.
		\item Type-F circuits, also called \textit{spanning circuits}, for $k \geq 3$, are the spanning forests of $G$ that are formed by $k-2$ trees.
	\end{enumerate}
\end{claim}

Before defining the stepwise circuit-cocircuit implementation we first need to introduce the following technical term. The \textit{level $i$ of recursion} in Algorithm \ref{alg:meta} is the number of calls to the procedure \textsc{GenCocircuits} currently being processed (on the \textit{call stack}).

\begin{defn}[Stepwise circuit-cocircuit implementation scheme]
	\label{defn:stepwise_scheme}

	We call an implementation of Algorithm \ref{alg:meta} \textit{stepwise} if, for every set $X = X_0$, $\lvert X_0 \rvert = l$, generated by the algorithm,

	\begin{enumerate}
		\item $X_0$ is an ordered sequence $(c_1, c_2,\ldots,c_l)$, where $c_i$ has been \linebreak added to $X_0$ at the level $i-1$ of recursion,
		\item there exists a mapping $s : \{1,2,\ldots,k\} \rightarrow \{0,1,\ldots,l\}$ such that $s(1) = 0$, $s(k) = l$ and, for each $j \in \{2,\ldots,k-1\}$, the set $\{c_1,\ldots,c_{s(j)}\} \subsetneq X_0$ forms a $j$-bond in $G$.
	\end{enumerate}

	\noindent For $j$, $1 \leq j < k$, we call the \textit{$j$-th stage of the algorithm} the steps the algorithm does at the levels $s(j), s(j) + 1,\ldots, s(j+1)-1$ of recursion. The algorithm selects elements $c_{s(j)+1},\ldots,c_{s(j+1)}$ in the $j$-th stage.

\end{defn}

\begin{defn}[Stepwise partition]
	\label{defn:stepwise_partition}
	Let $Z$ be a $k$-bond. For $1 \leq j < k$, let $Z^{j} = \{c_{s(j)+1},\ldots,c_{s(j+1)}\}$. An ordered partition $(Z^1, \ldots, Z^{k-1})$ of $Z$ is called a~stepwise partition if there exists a~stepwise implementation (confer Definition \ref{defn:stepwise_scheme}) of Algorithm \ref{alg:meta}, such that the elements of $Z^j$ are added to $X$ in its $j$-th stage.
\end{defn}

\begin{rem}
	There is a one to one correspondence between the stepwise partition and the mapping $s$.
\end{rem}

\begin{rem}
	Not every ordering of $\{Z^1, \ldots, Z^{k-1}\}$ is a valid stepwise partition. On the other hand, by Lemma \ref{lem:stepwise_is_welldefined}, there exists a valid stepwise partition for each $k$-bond.
\end{rem}

The following lemma asserts that the $k$-bonds can be generated by a stepwise implementation of the Circuit-Cocircuit algorithm.

\begin{lem}
	\label{lem:stepwise_is_welldefined}
	Let $G$ be a connected graph and $X \subseteq E(G)$ a $k$-bond, $k \geq 2$.
	For any $i$-bond $Z \subset X$ of $G$, $i < k$, there exists an $(i+1)$-bond $Y$ of $G$ such that $X \supseteq Y \supset Z$.
\end{lem}

\begin{proof}
	Let $A = X \sm Z$, $e \in A$ and $G'' = G \sm X$. $G'' \cup \{e\}$ has $k-1$ components, since $X$ is a minimal $k$-way cut. Choose maximal $A^1 \subseteq A$ with $e \in A^1$, such that $G'' \cup A^1$ has $k-1$ components. In the same manner construct $A^2,\ldots,A^{k-i-1}$, where $A^{k-i-1}$ is a maximal subset of $A$ such that ${G'' \cup A^{k-i-1}}$ has $i$ components. This constructions yields any $Z = X \sm A^{k-i-1}$.

	Now consider $G' = G \sm Z$, a~graph on $i$ components. By $B$ denote \linebreak ${A^{k-i-1} \sm A^{k-i-2}}$ and by $G'_0$ denote the component of $G'$ containing $B$. We choose some $f \in B$ and observe that all the remaining edges in $B$ (and no other edges in $G'_0$) lie on a common cycle with $f$ in $G'_0$. Thus $G' \sm B$ is a~graph on $i+1$ components and $Y = Z \cup B$.

	In the base case, $k = 2$, this construction is clearly possible and the lemma follows by induction.
\end{proof}

With Theorem \ref{meta_alg_correctness} and Lemma \ref{lem:stepwise_is_welldefined} we can easily conclude:

\begin{thm}
	\label{thm:stepwise_impl_correctness}
	For any given graph $G$ a stepwise implementation of Algorithm \ref{alg:meta} can be used to generate all the $k$-bonds in $G$.
\end{thm}

We will also use the following two technical lemmas.

\begin{lem}
	\label{stepwise_2bond}
	Let $G$ be a connected graph and $X \subseteq E(G)$ any $k$-bond, and for~$j$,~$1 \leq j \leq k$, denote by $X^j$ the set $\{c_1,\ldots,c_{s(j)}\}$ and by $G^j$ the graph $G \setminus X^{j}$. For $j$, $1 \leq j < k$, there is a connected component $G_0^j$ of $G^j$ such that $Z^j = X^{j+1} \setminus X^{j} \subseteq E(G_0^j)$ and $Z^j$ is a $2$-bond in $G_0^j$.
\end{lem}

\begin{proof}
	The set $B$ from the proof of Lemma \ref{lem:stepwise_is_welldefined} is a $2$-bond.
\end{proof}


\begin{lem}
	\label{lem:kway_circuit_types}
	For each stage $j$, $1 < j < k$, of a stepwise implementation of the Circuit-Cocircuit Meta-algorithm \ref{alg:meta} respecting the Definition \ref{defn:stepwise_scheme} holds the following,

	\begin{enumerate}[label=\alph*.]
		\item the set $C$ chosen on line 10 at the level $s(j)$ has to be a type-F (spanning) circuit,
		\item at the levels $i$, $s(j)+1 \leq i < s(j+1)$ of recursion, the set $C$ can always be chosen as a type-C circuit.
	\end{enumerate}
\end{lem}

\begin{proof}[Proof (\ref{lem:kway_circuit_types}.a)]
	In each stage $j$, the first edge the algoritm selects is $c_{s(j)+1}$ from $C$ which is a $k$-way circuit in $G$. Since the set $\{c_1,\ldots,c_{s(j)}\}$ forms a $j$-bond in $G$, the value of $C$ at the level $s(j)$ cannot be a type-C circuit and thus it has to be a type-F circuit by Claim \ref{claim:circuit_types}.
\end{proof}

\begin{proof}[Proof (\ref{lem:kway_circuit_types}.b)]
	By its construction the set $B$ in proof of Lemma \ref{lem:stepwise_is_welldefined} has exactly one edge, call it $f$, from some type-F circuit and each edge $e \in (B \setminus \{f\})$ lies in a type-C circuit together with $f$.
\end{proof}

\subsection*{Extending a $j$-bond to $(j+1)$-bonds}

To modify the original algorithm to fit into the scheme for generating $k$-bonds we solve the problem of extending a given $j$-bond $Y$ into a set of $(j+1)$-bonds, each of them a superset of $Y$.

\begin{algorithm}
	\caption{One stage of stepwise implementation}
	\label{alg:one-stage}
\begin{algorithmic}[1]
	\Require A connected graph $G$, parameters $j, k, m \in \mathbb{N}, j < k, m \geq 1$, and a $j$-bond $Y_1 \subseteq E(G)$
	\Ensure A collection of $(j+1)$-bonds such that for each $k$-bond $Y$, $Y_1 \subseteq Y \subseteq E(G)$, $\lvert Y \rvert \leq m$, some subset of $Y$ is among the generated $(j+1)$-bonds
	\If{$j=1$} \Comment{$Y_1 = \varnothing$, select a $k$-way basis}
		\State $F \leftarrow$ an arb. spanning forest of $k-1$ trees
		\Else \Comment{$Y_1 \neq \varnothing$, select a type-F circuit}
		\State $F \leftarrow$ an arb. spanning forest of $k-2$ trees and ${\lvert F \cap Y_1 \rvert = 1}$
	\EndIf
	\ForAll{$d = \{u, v\} \in F \setminus Y_1$}
		\State \Call{GenStage}{$j, Y_1, X = \{d\}, V_r = \{u\}, V_b = \{v\}, T_r = \{u\}$}
	\EndFor

	\Procedure{GenStage}{$j, Y, X, V_r, V_b, T_r$}
	\State Let $G_1 \subseteq G$ be the component of $G \setminus Y$ containing $X$
	\If{$\lvert Y \cup X \rvert > m - k + j + 1$}
		\State \textbf{return $\undefined$}
	\EndIf
	\If{there does not exist a connected subgraph \\
		\qquad \enspace $T_b \subseteq (G_1 \sm V(T_r)) \sm X$ such that $V_b \subsetneq V(T_b)$}
		\State \textbf{return $\undefined$}
	\EndIf
	\State $P \leftarrow$ a $V(T_r){-}V_b$ path $P \subseteq G_1$

	\If{such $P$ does not exist}
		\State \textbf{return} $Y \cup X$ \Comment{$Y \cup X$ is a $j+1$-bond}
	\Else
		\ForAll{$c \in P$}
			\State Let $u$ be the vertex in $c = \{u,v\}$ which is closer to $T_r$
			\State Let $P_u$ be the component of $P - c$ which contains $u$
			\State \Call{GenStage}{$j, Y, X \cup \{c\}, V_r \cup \{u\}, V_b \cup \{v\}, T_r \cup P_u$}
		\EndFor
	\EndIf

	\EndProcedure
\end{algorithmic}
\end{algorithm}

\clearpage

Due to Theorem \ref{thm:stepwise_impl_correctness} and the fact that Algorithm \ref{alg:one-stage} is just a special case of Algorithm \ref{alg:meta} (proven correct by Theorem \ref{meta_alg_correctness}) we can easily conclude Theorem \ref{thm:alg_one_stage_correctness}.

 \begin{thm}
	\label{thm:alg_one_stage_correctness}
	Let $G$ be a graph, $j,k,m$ natural numbers such that $j < k, m \geq 1$ and $Y_1 \subseteq E(G)$ a $j$-bond in $G$. Algorithm \ref{alg:one-stage} generates a set $\mathcal{S}$ of $(j+1)$-bonds such that for each $k$-bond $Y$, $Y_1 \subseteq Y \subseteq E(G)$, $\lvert Y \rvert \leq m$, some subset of $Y$ is among the generated $(j+1)$-bonds in $\mathcal{S}$.
\end{thm}



%\begin{proof}

%	It follows from Lemma \ref{lem:stepwise_is_welldefined}, that for the given $Y_1$ and each $k$-bond $Y$ such that $Y_1 \subseteq Y$, there is a $(j+1)$-bond $Y_2$ such that $Y_1 \subsetneq Y_2 \subseteq Y$. % We have to show that the algorithm returns this $Y_2 \in \mathcal{S}$.
%	To prove that the algorithm returns this $Y_2 \in \mathcal{S}$, it suffices to prove the correcntess of the choice of the first edge.

%	If $j = 1$, the first edge $c_1 = c_{s(1)+1} \in Y_2 \setminus Y_1$ has to be chosen from a $k$-way basis and the algorithm does this on line 2. If $j > 1$, by Lemma \ref{lem:kway_circuit_types}.a, the edge $c_{s(j)+1} \in Y_2 \setminus Y_1$ has to be chosen from a type-F circuit -- the algorithm does this choice on line 4.

% For both $j = 1$ and $j > 1$, when choosing the remaining edges of $Y_2 \setminus Y_1$ in the steps $s(j) + 1, \ldots, s(j+1)-1$, the algorithm selects them from a type-C circuit, which is possible by Lemma \ref{lem:kway_circuit_types}.b.


 % It remains to show that all $Y_2 \in \mathcal{S}$ such that $Y_1 \subsetneq Y_2 \subseteq Y$ with a certain $X_1 \subseteq Y_2 \setminus Y_1$ satisfy the size bound $\lvert Y_1 \cup X_1 \rvert \leq m - k + j + 1$. Suppose $\lvert Y_1 \cup X_1 \rvert > m - k + j + 1$. Thus in the next $k - j - 1$ stages the algorithm has to select less than $k - j - 1$ edges in the remaining steps to satisfy $\lvert Y \rvert \leq m$, contradicting our assumption that $Y$ is a $k$-bond.
% \end{proof}
