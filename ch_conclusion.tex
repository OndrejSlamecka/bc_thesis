
\chapter{Conclusion}

We presented in detail the circuit-cocircuit algorithm together with its application to $k$-way cuts in a graph. We further improved the algorithm to generate the cuts almost canonically. Several aspects of practical implementation of the algorithm were described and finally we presented results of evaluation of our implementation. We aimed to make the program distributed with this thesis to be of a good readability while maintaining a good execution speed.

In conclusion our implementation solves the problem of finding all small multiway cuts correctly as well as quickly (given the complexity of the problem), thus demonstrating the feasibility of this algorithm for practical computations, say for use by infrastructure planners. In particular the algorithm performs significantly better than the "brute-force" algorithm when the cut size bound is greater than the multiplicity of the cuts.

We hope that this thesis provides a good basis for possible future extensions. The main question is whether there exists a method of totally canonical generation which does not require explicit isomorphism checking. The results from Chapter \ref{chapter_evaluation} should serve well when paralellizing the algorithm. Moreover, as that chapter suggest the algorithm spends the most of the time in \lstinline|shortestPath| procedure -- a~CPU-aware implementation \cite{parallel_bfs} of this procedure would benefit the running time.
