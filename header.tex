\documentclass[12pt,oneside]{fithesis2}
\usepackage[english]{babel}       % Multilingual support
\usepackage[utf8]{inputenc}       % UTF-8 encoding
\usepackage[T1]{fontenc}          % T1 font encoding
\usepackage[                      % A sans serif font that blends well with Palatino
  scaled=0.86
]{berasans}
\usepackage[                      % A tt font if you do not like LM's tt
  scaled=1.03
]{inconsolata}
\usepackage[usenames,dvipsnames]{xcolor}
\usepackage[                      % Clickable links
  plainpages = false,             % We have multiple page numberings
  pdfpagelabels                   % Generate pdf page labels
]{hyperref}

\hypersetup{
	pdftitle={Efficient exhaustive generation of small multiway cuts},
	pdfauthor={Ondřej Slámečka},
	colorlinks=true,     % uses boxes if set to false
	linkcolor=BrickRed,  % color of internal links
	citecolor=BrickRed,  % color of links to bibliography
	filecolor=BrickRed,  % color of file links
	urlcolor=BrickRed    % color of external links
}

\usepackage[backend=biber, sorting=none]{biblatex}
\addbibresource{thesis.bib}
%\usepackage{showframe}

\usepackage{tikz}
\usepackage{enumitem}
\usepackage{mathtools}
\usepackage{amssymb}
\usepackage{relsize} % mathlarger
\usepackage{amsthm}
\usepackage{algorithm, algpseudocode}
\usepackage{complexity} % #P

\renewcommand{\algorithmicrequire}{\textbf{Input:}}
\renewcommand{\algorithmicensure}{\textbf{Output:}}
\algnewcommand{\algorithmicendif}{\textbf{fi}}
\algblockdefx[IF]{If}{EndIf}[1]{\algorithmicif\ #1\ \algorithmicthen}{\algorithmicendif}

\makeatletter
\renewcommand\thealgorithm{\thechapter.\arabic{algorithm}}
\@addtoreset{algorithm}{chapter}
\makeatother

\usepackage{listings}
\lstset{basicstyle=\ttfamily\small}

\DeclareMathOperator{\mtime}{time}
\DeclareMathOperator{\mbonds}{bonds}
\DeclareMathOperator{\mlen}{len}
\DeclareMathOperator{\mcan}{can}

% amsthm
\theoremstyle{plain}% default
\newtheorem{thm}[algorithm]{Theorem}
\newtheorem{lem}[algorithm]{Lemma}
\newtheorem{prop}[algorithm]{Proposition}
\newtheorem{cor}[algorithm]{Corollary}

\theoremstyle{definition}
\newtheorem{defn}[algorithm]{Definition}
\newtheorem{claim}[algorithm]{Claim}
\newtheorem{exmp}[algorithm]{Example}

\theoremstyle{remark}
\newtheorem*{rec}{Recall}
\newtheorem*{rem}{Remark}
\newtheorem*{note}{Note}

% tables
\usepackage{longtable}%
\usepackage{colortbl}%
\newcommand{\evenrowcolor}{\rowcolor[gray]{0.925}}

\usepackage{booktabs}
%\usepackage{slashbox}

% gnuplottex
\usepackage{gnuplottex}

% section line
\newcommand{\sectionline}{%
	\nointerlineskip \vspace{0.2cm}%
	\begin{center}
		\rule{0.5\linewidth}{.7pt}
	\end{center}
	\par\nointerlineskip \vspace{0.2cm} % \baselineskip
}


% http://tex.stackexchange.com/questions/89745/how-to-diagonally-divide-a-table-cell-properly
\usepackage{array}
\usepackage{makecell}
\newcolumntype{x}[1]{>{\centering\arraybackslash}p{#1}}
\usepackage{tikz}
\newcommand\diag[4]{%
  \multicolumn{1}{p{#2}|}{\hskip-\tabcolsep
  $\vcenter{\begin{tikzpicture}[baseline=0,anchor=south west,inner sep=#1]
  \path[use as bounding box] (0,0) rectangle (#2+2\tabcolsep,\baselineskip);
  \node[minimum width={#2+2\tabcolsep-\pgflinewidth},
        minimum  height=\baselineskip+\extrarowheight-\pgflinewidth] (box) {};
  \draw[line cap=round] (box.north west) -- (box.south east);
  \node[anchor=south west] at (box.south west) {#3};
  \node[anchor=north east] at (box.north east) {#4};
 \end{tikzpicture}}$\hskip-\tabcolsep}}

% table and figures should share counters
\makeatletter
\renewcommand*{\thetable}{\arabic{chapter}.\arabic{table}}
\renewcommand*{\thefigure}{\arabic{chapter}.\arabic{figure}}
\let\c@table\c@figure
\makeatother 

