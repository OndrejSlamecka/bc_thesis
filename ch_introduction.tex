
\chapter{Introduction}

Graphs consisting of nodes and edges can be used to model networks, for example road networks (cities and road junctions form nodes and the roads connecting them form edges) or computer networks. Naturally, infrastructure planners try to design a structure which prevents failures of the network. Failures may be of various kind but we shall focus on failures caused by interruption of the edges. Our task is to engineer an algorithm which would help identify all the sets of edges which, although small in number of edges, cause a (possibly multiple) disconnection of the graph.

In the language of graph theory we are looking for all small multiway cuts. An easy reduction from the problem of enumerating all cuts with minimum cardinality to the problem of enumerating all multiway cuts can be shown, proving the non-parametrized version of our problem is $\#\P$-complete \cite{Provan1983}. Although some related research was done \cite{cactus}, no one seems to have focused on this exact implementation problem except Bíl et~al.\ in \cite{cdv}. They use a predecessor of the algorithm described in this thesis, both devised by the same author.

In this thesis we show a new algorithm proposed by Petr Hliněný \cite{hlineny_circuitcocircuit}. The central idea of this algorithm is to approach the problem with a more general perspective of \textit{matroid} and find all \textit{cocircuits} of this matroid. The algorithm is thus called \textit{Circuit-Cocircuit}. Our aim is to implement the algorithm and show its applicability to practical data.

Chapter 2 with the basic definitions and terminology follows after this introduction. In Chapter 3 we first state the abstract circuit-cocircuit algorithm for matroids and then propose modifications to generate first $2$-way minimal edge cuts and then $k$-way edge cuts. In Chapter 4 a scheme for canonical generation is provided. We give finer details about the practical implementation of this algorithm in Chapter 5. In Chapter 6 we present results of measurements of different properties of the algorithm and Chapter~7 provides conclusion and suggestions for future work.

